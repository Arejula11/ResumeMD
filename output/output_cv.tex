\documentclass[10pt, letterpaper]{article}

% Packages:
\usepackage[
    ignoreheadfoot, % set margins without considering header and footer
    top=1 cm, % seperation between body and page edge from the top
    bottom=1 cm, % seperation between body and page edge from the bottom
    left=2 cm, % seperation between body and page edge from the left
    right=2 cm, % seperation between body and page edge from the right
    footskip=1.0 cm, % seperation between body and footer
    % showframe % for debugging 
]{geometry} % for adjusting page geometry
\usepackage{titlesec} % for customizing section titles
\usepackage{tabularx} % for making tables with fixed width columns
\usepackage{array} % tabularx requires this
\usepackage[dvipsnames]{xcolor} % for coloring text
\definecolor{primaryColor}{RGB}{0, 0, 0} % define primary color
\usepackage{enumitem} % for customizing lists
\usepackage{fontawesome5} % for using icons
\usepackage{amsmath} % for math
\usepackage[
    pdftitle={Miguel Aréjula CV},
    pdfauthor={Miguel Aréjula},
    pdfcreator={LaTeX with RenderCV},
    colorlinks=true,
    urlcolor=primaryColor
]{hyperref} % for links, metadata and bookmarks
\usepackage[pscoord]{eso-pic} % for floating text on the page
\usepackage{calc} % for calculating lengths
\usepackage{bookmark} % for bookmarks
\usepackage{lastpage} % for getting the total number of pages
\usepackage{changepage} % for one column entries (adjustwidth environment)
\usepackage{paracol} % for two and three column entries
\usepackage{ifthen} % for conditional statements
\usepackage{needspace} % for avoiding page brake right after the section title
\usepackage{iftex} % check if engine is pdflatex, xetex or luatex

% Ensure that generate pdf is machine readable/ATS parsable:
\ifPDFTeX
    \input{glyphtounicode}
    \pdfgentounicode=1
    \usepackage[T1]{fontenc}
    \usepackage[utf8]{inputenc}
    \usepackage{lmodern}
\fi

\usepackage{charter}

% Some settings:
\raggedright
\AtBeginEnvironment{adjustwidth}{\partopsep0pt} % remove space before adjustwidth environment
\pagestyle{empty} % no header or footer
\setcounter{secnumdepth}{0} % no section numbering
\setlength{\parindent}{0pt} % no indentation
\setlength{\topskip}{0pt} % no top skip
\setlength{\columnsep}{0.15cm} % set column seperation
\pagenumbering{gobble} % no page numbering

\titleformat{\section}{\needspace{4\baselineskip}\bfseries\large}{}{0pt}{}[\vspace{1pt}\titlerule]

\titlespacing{\section}{
    % left space:
    -1pt
}{
    % top space:
    0.1 cm
}{
    % bottom space:
    0.2 cm
} % section title spacing

\renewcommand\labelitemi{$\vcenter{\hbox{\small$\bullet$}}$} % custom bullet points
\newenvironment{highlights}{
    \begin{itemize}[
        topsep=0.10 cm,
        parsep=0.10 cm,
        partopsep=0pt,
        itemsep=0pt,
        leftmargin=0 cm + 10pt
    ]
}{
    \end{itemize}
} % new environment for highlights


\newenvironment{highlightsforbulletentries}{
    \begin{itemize}[
        topsep=0.10 cm,
        parsep=0.10 cm,
        partopsep=0pt,
        itemsep=0pt,
        leftmargin=10pt
    ]
}{
    \end{itemize}
} % new environment for highlights for bullet entries

\newenvironment{onecolentry}{
    \begin{adjustwidth}{
        0 cm + 0.00001 cm
    }{
        0 cm + 0.00001 cm
    }
}{
    \end{adjustwidth}
} % new environment for one column entries

\newenvironment{twocolentry}[2][]{
    \onecolentry
    \def\secondColumn{#2}
    \setcolumnwidth{\fill, 4.5 cm}
    \begin{paracol}{2}
}{
    \switchcolumn \raggedleft \secondColumn
    \end{paracol}
    \endonecolentry
} % new environment for two column entries

\newenvironment{threecolentry}[3][]{
    \onecolentry
    \def\thirdColumn{#3}
    \setcolumnwidth{, \fill, 4.5 cm}
    \begin{paracol}{3}
    {\raggedright #2} \switchcolumn
}{
    \switchcolumn \raggedleft \thirdColumn
    \end{paracol}
    \endonecolentry
} % new environment for three column entries

\newenvironment{header}{
    \setlength{\topsep}{0pt}\par\kern\topsep\centering\linespread{1.5}
}{
    \par\kern\topsep
} % new environment for the header

\newcommand{\placelastupdatedtext}{% \placetextbox{<horizontal pos>}{<vertical pos>}{<stuff>}
  \AddToShipoutPictureFG*{% Add <stuff> to current page foreground
    \put(
        \LenToUnit{\paperwidth-2 cm-0 cm+0.05cm},
        \LenToUnit{\paperheight-1.0 cm}
    ){\vtop{{\null}\makebox[0pt][c]{
        \small\color{gray}\textit{Last updated in September 2024}\hspace{\widthof{Last updated in September 2024}}
    }}}%
  }%
}%

% save the original href command in a new command:
\let\hrefWithoutArrow\href

% new command for external links:

%--------------------------------------------------------------------------------------------------------------
%--------------------------------------------------------------------------------------------------------------
%--------------------------------------------------------------------------------------------------------------

\begin{document}
    \newcommand{\AND}{\unskip
        \cleaders\copy\ANDbox\hskip\wd\ANDbox
        \ignorespaces
    }
    \newsavebox\ANDbox
    \sbox\ANDbox{$|$}

    \begin{header}
    \fontsize{25 pt}{25 pt}\selectfont Miguel Aréjula Aísa

\vspace{5 pt}
\normalsize
\mbox{\hrefWithoutArrow{mailto:arejula10@gmail.com}{\texttt{arejula10@gmail.com}}}%
\kern 5.0 pt%
\AND%
\kern 5.0 pt%
\mbox{\hrefWithoutArrow{tel:+34 601 49 10 89}{+34 601 49 10 89}}%
\kern 5.0 pt%
\AND%
\kern 5.0 pt%
\mbox{\hrefWithoutArrow{https://www.linkedin.com/in/miguel-arejula-aisa-653088291}{\texttt{LinkedIn}}}%
\kern 5.0 pt%
\AND%
\kern 5.0 pt%
\mbox{\hrefWithoutArrow{https://github.com/Arejula11}{\texttt{GitHub}}}%
\kern 5.0 pt%
\AND%
\kern 5.0 pt%
\mbox{\hrefWithoutArrow{https://www.are-dev.es}{\texttt{Blog}}}%

    \end{header}

%--------------------------------------------------------------------------------------------------------------
%--------------------------------------------------------------------------------------------------------------
%--------------------------------------------------------------------------------------------------------------
    \vspace{5 pt - 0.3 cm}
    \section{Education}
    \begin{twocolentry}{Sept 2025 – June 2027}
\textbf{Master in Software Engineering}, University of Southern Denmark
\end{twocolentry}

\vspace{0.10 cm}
\begin{onecolentry}
\begin{highlights}
\item \textbf{Coursework:} Advanced Software Engineering Methodologies, Advanced Software Architecture and Analysis Techniques, Big Data and Data Science Technology
\end{highlights}
\end{onecolentry}
\vspace{0.20 cm}
\begin{twocolentry}{Sept 2021 – June 2025}
\textbf{Bachelor in Software Engineering}, University of Zaragoza
\end{twocolentry}

\vspace{0.10 cm}
\begin{onecolentry}
\begin{highlights}
\item \textbf{Coursework:} Software Engineer, Distributed Systems, Software Architecture, Prerequisite Engineering, Verification and Validation, Agile Methodologies and Quality, and Artificial Intelligence.
\item Achieved high honors in Information Systems II and Verification and Validation, demonstrating exceptional proficiency and understanding in these subjects.
\end{highlights}
\end{onecolentry}
\vspace{0.20 cm}
\begin{twocolentry}{July 2024}
\textbf{Cambridge English  Advance}, 
\end{twocolentry}


%--------------------------------------------------------------------------------------------------------------
%--------------------------------------------------------------------------------------------------------------
%--------------------------------------------------------------------------------------------------------------
    \section{Experience}

    \begin{twocolentry}{June 2024 – August 2025}
\textbf{Software Engineer}, University of Zaragoza
\end{twocolentry}

\vspace{0.10 cm}
\begin{onecolentry}
\begin{highlights}
\item Researcher in project TED2021-130449B-I00 at the University of Zaragoza, where I led development of a custom web application for the Traumatology Department (Hospital Clínico Lozano Blesa), from requirements gathering with medical staff to architecture design and final delivery.
\item Developed a full-stack system using React and PostgreSQL, with RESTful APIs in Express and FastAPI, streamlining operations and expected to support ~50 patients per day, enhancing efficiency and quality of care.
\item Integrated and processed structured clinical data (CSV-based surgical records) into a relational model, enabling generation and validation of clinical pathways through Petri nets and formal methods.
\end{highlights}
\end{onecolentry}
\vspace{0.10 cm}
\begin{twocolentry}{Starting February 2026}
\textbf{Teaching Assistant (Incoming)}, SDU
\end{twocolentry}

\vspace{0.10 cm}
\begin{onecolentry}
\begin{highlights}
\item Support students in understanding core software architecture concepts by assisting with lab sessions, assignments, and problem-solving, while clearly explaining technical concepts and collaborating with professors.
\end{highlights}
\end{onecolentry}


%--------------------------------------------------------------------------------------------------------------
%--------------------------------------------------------------------------------------------------------------
%--------------------------------------------------------------------------------------------------------------


\section{Projects}
    \vspace{0.10 cm}
\begin{twocolentry}{\href{are-dev.es}{are-dev.es}}\textbf{Are-Dev}
\end{twocolentry}

\vspace{0.10 cm}
\begin{onecolentry}
\begin{highlights}
\item Are-dev is a personal technical blog and YouTube channel focused on software development, front-end technologies, and modern frameworks.
\item Publish blog posts and videos, sharing tutorials, project walk-throughs, and insights to engage the developer community. \textbf{Tools Used:} Astro, Vercel, Markdown.
\end{highlights}
\end{onecolentry}
\vspace{0.10 cm}
\begin{twocolentry}{\href{https://github.com/The-European-Avengers/pizza-i4-architecture-group2}{github.com/I4-Pizza}}\textbf{I4 Pizza Production System}
\end{twocolentry}

\vspace{0.10 cm}
\begin{onecolentry}
\begin{highlights}
\item Industry 4.0 pizza production system integrating warehouse, production line, and web platform via a distributed architecture.
\item Designed, validated, and documented the system architecture using ADD, covering requirements, use cases, microservice boundaries, formal verification with UPPAAL, and experimental evaluation.
\item Led system-level coordination, ensured architectural consistency, and supervised implementation. \textbf{Tools Used:} Go, Python, Kafka, Docker, UPPAAL.
\end{highlights}
\end{onecolentry}
\vspace{0.10 cm}
\begin{twocolentry}{\href{https://github.com/STW-24-25}{github.com/AgroNet}}\textbf{AgroNet}
\end{twocolentry}

\vspace{0.10 cm}
\begin{onecolentry}
\begin{highlights}
\item Designed and developed a full-stack collaborative web platform for farmers, integrating real-time market prices, personalized weather alerts, and interactive geospatial data visualization across Spain.
\item Led the front-end team, developed the web application, and deployed it on Vercel.
\item Managed communication with the backend hosted on AWS, ensuring reliable API integration and real-time data synchronization. \textbf{Tools Used:} Astro, NodeJs, TypeScript, MongoDB, AWS, Vercel.
\end{highlights}
\end{onecolentry}
\vspace{0.10 cm}
\begin{twocolentry}{\href{https://github.com/The-European-Avengers/BigDataProject}{github.com/BigDataProject}}\textbf{Energy Price Prediction}
\end{twocolentry}

\vspace{0.10 cm}
\begin{onecolentry}
\begin{highlights}
\item Using Danish Meteorological Institute (DMI) weather datasets and national energy consumption/price data to analyze correlations and forecast future electricity prices.
\item Expected insights include: renewable energy impact, peak price periods, seasonal patterns, and cost optimization windows.
\item Applying big data frameworks and machine learning models to process large-scale datasets and deliver predictive analytics. \textbf{Tools Used:} Kafka, HDFS, Kubernetes, Hive, Spark, Python.
\end{highlights}
\end{onecolentry}



\end{document}